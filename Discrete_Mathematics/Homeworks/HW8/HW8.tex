\documentclass[a4paper, 12pt]{article}
\usepackage[utf8]{inputenc}
\usepackage[english, russian]{babel}
\usepackage{amsmath}
\usepackage[left=2cm,right=2cm, top=2cm,bottom=2cm]{geometry}

\begin{document}

\begin{flushright}
  Бавыкин Роман Р3110
\end{flushright}

\begin{center}
  Домашнее задание 8\\
  ДЕЛЕНИЕ ЧИСЕЛ С ПЛАВАЮЩЕЙ ЗАПЯТОЙ\\
  Вариант 6
\end{center}
$A=1,3;\ B=0,018.$
\begin{enumerate}
  \item Деление в формате Ф1.\\
  $A=(1,3)_{10}=(1,4C)_{16}=(0,14С)_{16}\cdot16^1$\\
  $B=(0,018)_{10}=(0,049B)_{16}=(0,49B)_{16}\cdot16^{-1}$\\
  $X_C=X_A-X_B+d=66;\ P_C=2$\\
  \begin{tabular}{|c|c|l|l|}
  	\hline
  	№ шага & Действие         & Делимое           & Частное \\
  	\hline
  	& $M_A$                   & 0 0 0 0 1 0 1 0 1 & 0 0 0 0 0 0 0 0 \\
  	0 & $[-M_B]_{\mbox{доп}}$ & 1 1 0 1 1 0 1 1 0 & \\
  	& $R_0$                   & 1 1 1 0 0 1 0 1 1 & 0 0 0 0 0 0 0 0 \\
  	\hline
  	& $\leftarrow R_0$        & 1 1 0 0 1 0 1 1 0 & 0 0 0 0 0 0 0 0 \\
  	1 & $[M_B]_{\mbox{пр}}$   & 0 0 1 0 0 1 0 1 0 & \\
  	& $R_1$                   & 1 1 1 1 0 0 0 0 0 & 0 0 0 0 0 0 0 0 \\
  	\hline
  	& $\leftarrow R_1$        & 1 1 1 0 0 0 0 0 0 & 0 0 0 0 0 0 0 0 \\
  	2 & $[M_B]_{\mbox{пр}}$   & 0 0 1 0 0 1 0 1 0 & \\
  	& $R_2$                   & 0 0 0 0 0 1 0 1 0 & 0 0 0 0 0 0 0 1 \\
  	\hline
  	& $\leftarrow R_2$        & 0 0 0 0 1 0 1 0 0 & 0 0 0 0 0 0 1 0 \\
  	3 & $[-M_B]_{\mbox{доп}}$ & 1 1 0 1 1 0 1 1 0 & \\
  	& $R_3$                   & 1 1 1 0 0 1 0 1 0 & 0 0 0 0 0 0 1 0 \\
  	\hline
  	& $\leftarrow R_3$        & 1 1 0 0 1 0 1 0 0 & 0 0 0 0 0 1 0 0 \\
  	4 & $[M_B]_{\mbox{пр}}$   & 0 0 1 0 0 1 0 1 0 & \\
  	& $R_4$                   & 1 1 1 0 1 1 1 1 0 & 0 0 0 0 0 1 0 0 \\
  	\hline
  	& $\leftarrow R_4$        & 1 1 0 1 1 1 1 0 0 & 0 0 0 0 1 0 0 0 \\
  	5 & $[M_B]_{\mbox{пр}}$   & 0 0 1 0 0 1 0 1 0 & \\
  	& $R_5$                   & 0 0 0 0 0 0 1 1 0 & 0 0 0 0 1 0 0 1 \\
  	\hline
  	& $\leftarrow R_5$        & 0 0 0 0 0 1 1 0 0 & 0 0 0 1 0 0 1 0 \\
  	6 & $[-M_B]_{\mbox{доп}}$ & 1 1 0 1 1 0 1 1 0 & \\
  	& $R_6$                   & 1 1 1 0 0 0 0 1 0 & 0 0 0 1 0 0 1 0 \\
  	\hline
  	& $\leftarrow R_6$        & 1 1 0 0 0 0 1 0 0 & 0 0 1 0 0 1 0 0 \\
  	7 & $[M_B]_{\mbox{пр}}$   & 0 0 1 0 0 1 0 1 0 & \\
  	& $R_7$                   & 1 1 1 0 0 1 1 1 0 & 0 0 1 0 0 1 0 0 \\
  	\hline
  	& $\leftarrow R_7$        & 1 1 0 0 1 1 1 0 0 & 0 1 0 0 1 0 0 0 \\
  	8 & $[M_B]_{\mbox{пр}}$   & 0 0 1 0 0 1 0 1 0 & \\
  	& $R_8$                   & 1 1 1 1 0 0 1 1 0 & 0 1 0 0 1 0 0 0 \\
  	\hline
  \end{tabular}\\
  $C^*=(0,48)_{16}\cdot16^2=(24)_{16}=72$\\
  $\Delta C=C_T-C^*=72,(2)-72=0,(2)$\\
  $\delta C=\left|\frac{\Delta C}{C_T}\right|\cdot100\%=0,3\%$\\
  Погрешность вызвана неточным представлением операндов.
  \newpage
  \item Деление в формате Ф2.\\
  $A=(1,3)_{10}=(1,0100110)_2=(0,10100110)_2\cdot2^1$\\
  $B=(0,018)_{10}=(0,0000010010011)_2=(0,10010011)_2\cdot2^{-5}$\\
  $X_C=X_A-X_B+d=132;\ P_C=6$\\
  \begin{tabular}{|c|c|l|l|}
  	\hline
  	№ шага & Действие         & Делимое           & Частное \\
  	\hline
  	& $M_A$                   & 0 1 0 1 0 0 1 1 0 & 0 0 0 0 0 0 0 0 \\
  	0 & $[-M_B]_{\mbox{доп}}$ & 1 0 1 1 0 1 1 0 1 & \\
  	& $R_0$                   & 0 0 0 0 1 0 0 1 1 & 0 0 0 0 0 0 0 1 \\
  	\hline
  	& $\leftarrow R_0$        & 0 0 0 1 0 0 1 1 0 & 0 0 0 0 0 0 1 0 \\
  	1 & $[-M_B]_{\mbox{доп}}$ & 1 0 1 1 0 1 1 0 1 & \\
  	& $R_1$                   & 1 1 0 0 1 0 0 1 1 & 0 0 0 0 0 0 1 0 \\
  	\hline
  	& $\leftarrow R_1$        & 1 0 0 1 0 0 1 1 0 & 0 0 0 0 0 1 0 0 \\
  	2 & $[M_B]_{\mbox{пр}}$   & 0 1 0 0 1 0 0 1 1 & \\
  	& $R_2$                   & 1 1 0 1 1 1 0 0 1 & 0 0 0 0 0 1 0 0 \\
  	\hline
  	& $\leftarrow R_2$        & 1 0 1 1 1 0 0 1 0 & 0 0 0 0 1 0 0 0 \\
  	3 & $[M_B]_{\mbox{пр}}$   & 0 1 0 0 1 0 0 1 1 & \\
  	& $R_3$                   & 0 0 0 0 0 0 1 0 1 & 0 0 0 0 1 0 0 1 \\
  	\hline
  	& $\leftarrow R_3$        & 0 0 0 0 0 1 0 1 0 & 0 0 0 1 0 0 1 0 \\
  	4 & $[-M_B]_{\mbox{доп}}$ & 1 0 1 1 0 1 1 0 1 & \\
  	& $R_4$                   & 1 0 1 1 1 0 1 1 1 & 0 0 0 1 0 0 1 0 \\
  	\hline
  	& $\leftarrow R_4$        & 0 1 1 1 0 1 1 1 0 & 0 0 1 0 0 1 0 0 \\
  	5 & $[M_B]_{\mbox{пр}}$   & 0 1 0 0 1 0 0 1 1 & \\
  	& $R_5$                   & 1 1 0 0 0 0 0 0 1 & 0 0 1 0 0 1 0 0 \\
  	\hline
  	& $\leftarrow R_5$        & 1 0 0 0 0 0 0 1 0 & 0 1 0 0 1 0 0 0 \\
  	6 & $[M_B]_{\mbox{пр}}$   & 0 1 0 0 1 0 0 1 1 & \\
  	& $R_6$                   & 1 1 0 0 1 0 1 0 1 & 0 1 0 0 1 0 0 0 \\
  	\hline
  	& $\leftarrow R_6$        & 1 0 0 1 0 1 0 1 0 & 1 0 0 1 0 0 0 0 \\
  	7 & $[M_B]_{\mbox{пр}}$   & 0 1 0 0 1 0 0 1 1 & \\
  	& $R_7$                   & 1 1 0 1 1 1 1 0 1 & 1 0 0 1 0 0 0 0 \\
  	& $M_C\rightarrow$        &                   & 0 1 0 0 1 0 0 0 0 \\
  	\hline
  \end{tabular}\\
  $C^*=(0.10010000)_2\cdot 2^7=(1001000)_2=72$\\
  $\Delta C=C_T-C^*=72,(2)-72=0,(2)$\\
  $\delta C=\left|\frac{\Delta C}{C_T}\right|\cdot100\%=0,3\%$\\
  Погрешность вызвана неточным представлением операндов.\\
  Получившиеся значения в форматах Ф1 и Ф2 совпадают, поэтому и погрешности совпадают. Но чаще всего результат в формате Ф2 будет точнее, так как операнды в этом формате представляются точнее.
\end{enumerate}
\end{document}
