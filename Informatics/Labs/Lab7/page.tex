\newpage
\begin{multicols}{2}
    \noindent\begin{tabular}{c c c c}
        \hline
        \textnumero \\ итерации & $x_1$ & $x_2$ & $x_3$ \\
        \hline
        1 & 1,33 & 0,73 & 0,38 \\
        2 & 1,06 & 0,75 & 0,47 \\
        3 & 0,977 & 0,78 & 0,498 \\
        4 & 0,954 & 0,797 & 0,499 \\
        5 & 0,95 & 0,799 & 0,499 \\
        \hline
    \end{tabular}
    
    Но вычисления на компьютере отличаются от ручных еще и тем, что слишком маленькие числа заменяются нулем, а слишком большие прерывают работу программы. Из-за этого вычисления методом Лобачевского нередко плохо кончаются. Если $|a_n|<0,1$, то уже после 7 итераций становится $a_n=0$, и далее считать невозможно. Если же $|a_n|>10$, то после 7 итераций станет $a_n>10^100$, и работа программы прервется. Для работы с этим надо с самого начала разделить все коэффициенты уравнения (1) на $a_n$ и таким образом сделать этот коэффициент равным единице. Но остается вторая трудность. Если $|x_1|<0,1$ или $|x_1|>10$, то такая же неприятность произойдет с коэффициентом  $a_{n-1}$, и получить значение $|x_1|$ станет невозможно.
 
    Чтобы улучшить алгоритм, можно сделать старший корень близким к единице с помощью подстановки $x=gX$, где $g$ --- приближенное значение этого корня. Для этого надо после, скажем, $p$ итераций (мысленно) сделать в уравнении (1) подстановку $x=gX, \mbox{ где } g=\\=d^r, d=-b_{n-1}/b_n, r=1/2^p.$ При этом и в очередном уравнении (3) для Q(t) надо сделать подстановку $$t=x^{2^p}=g^{2^p}*X^{2^p}=dT.$$ а его коэффициенты $b_i$(уже не мысленно) надо заменить на $b_i/d^{n-i}$. После этого мы можем продолжать вычисления, но нужно будет накапливать произвведение $g_1g_2\ldots$ значений чисел $g=d^r$ на очередной итерации, где производилась подстановка, и дмоножать на него получаемые значения корней $x_k.$
 
    3. Упражнения для вычисления корней многочленов
 
    а) $24-50x+35x^2-10x^3+x^4=0;\\ \mbox{ корни: } 1,\ 2,\ 3,\ 4.$
 
    б) $-63,84+124,48x-73,36x^2-\\-4,88x^3+21,61x^4-8,2x^5+x^6=0;\\ \mbox{ корни: } 2,1;\ 2;\ 2;\ 2;\ 2;\ - 1,9.$
 
    в) $90+19x+x^2=0; \mbox{ корни: }-9,\\ -10.$
 
    г) $0,009-0,19x+x^2=0; \mbox{ корни: } 0,1;\\ 0,09.$
 
    д) $-0,33264+0,4278x+1,129x^2-\\-1,35x^3-0,9x^4+x^5=0; \mbox{ корни: } 1,1;\\\ -0,9;\ 0,8;\ -0,7;\ 0,6.$
 \\
 
    Дополнительные вопросы
    \begin{enumerate}
        {\item Как преобразовать уравнение (2), чтобы минимальный (по модулю) корень превратить в максимальный?}
        {\item Тот же вопрос для корня, находящегося вблизи заданного значения  $x_0.$}
    \end{enumerate}
\end{multicols}
\includegraphics[scale=0.7]{smile.png}
\begin{multicols}{3}
    \subsection*{Как не слушать оратора}
    \small Ни один оратор, какова бы ни была его энергия, не имеет шансов победить сопливость слушателей... Немногие из нас имеют мужество спать открыто и честно во время официальной речи. После тщательного исследования этого вопроса я могу представить на рассмотрение читателя несколько оригинальных методов, которые до сих пор не публиковались.
    
    Усядтесь в кресло как можно глубже, голову склоните слега вперед (это освобождает язык, он висит свободно, не затрудняя дыхание). Громкий храп выводит из себя даже самого смиренного оратора, поэтому главное --- избегайте храпа, все дыхательные пути должны быть свободны. Трудно дать четкие инструкции по сохранению во сне равновесия. Но чтобы голова не моталась из стороны в сторону, устройте ей из двух рук и туловища прочную опору в форме треножника... Так у вас и голова не упадет на грудь, и челюсть не отвалится. Закрытые глаза следует прятать в ладонях, при этом пальцы должны сжимать лоб в гармошку. Это производит впечатление напряженной работы мысли и несколько озадачивает оратора. Возможны выкрики во время кошмаров, но на этот риск придется идти. Просыпайтесь медленно, оглянитесь и не начинайте аплодировать сразу. Это может оказаться невпопад. Лучше уж подождите, пока вас разбудят заключительные аплодисменты.
    \begin{flushright}
        {\textit{У.Б. Бин\\(Из книги "Физики\\продолжают шутить"}}
    \end{flushright}
\end{multicols}

