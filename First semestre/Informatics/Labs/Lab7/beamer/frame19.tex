\begin{frame}[t]{\textcolor{blue}{П}ример использования меры Шеннона}
	\noindent Шулер наугад вытаскивает одну карту из стопки, содержащей 9 известных ему карт: 3 джокера, 3 туза, 1 король, 1 дама и 1 валет. Какое количество информации для шулера содержится в этом событии s?
	$$\mbox{Вероятность вытащить} \left\{
	\begin{array}{l}
		\mbox{джокера}\\
		\mbox{туза}\\
		\mbox{короля}\\
		\mbox{даму}\\
		\mbox{валета}
	\end{array}
	\right\}\mbox{равна}\left\{
	\begin{array}{l}
		3/9=1/3\\
		3/9=1/3\\
		1/9\\
		1/9\\
		1/9\\
	\end{array}	
	\right .$$
	Количество информации, выраженное в тритах, равно:
		$$i(s)=-\left(\frac{1}{3}\cdot\log_3\frac{1}{3}+\frac{1}{3}\cdot\log_3\frac{1}{3}+\frac{1}{9}\cdot\log_3\frac{1}{9}+\frac{1}{9}\cdot\log_3\frac{1}{9}+\frac{1}{9}\cdot\log_3\frac{1}{9}\right)=$$
		$$=\frac{1}{3}+\frac{1}{3}+\frac{2}{9}+\frac{2}{9}+\frac{2}{9}=1+\frac{1}{3}\approx\log_35vs\log_314$$
\end{frame}